\documentclass{article}

\input{0. Packages}
\newdateformat{monthyeardate}{%
  \monthname[\THEMONTH], \THEYEAR}


\title{Ice \& Climate \\ \vspace{1em} \large Assignment 1: Surface Energy Budget \normalsize}
\author{Yvo Werner\\ 9090649 } 
\date{September 2024}

\renewcommand{\arraystretch}{1.25}

\hfuzz=200pt



\begin{document}


\maketitle

\section*{Part a)}
% Which component, shortwave or longwave, would you expect to dominate the all-wave cloud effect over dark surfaces, and what does this imply for the sign of the total cloud effect over dark surfaces?



\section*{Part b)}
% Explain the global patterns. The positive total cloud effect in some parts of the Polar Regions is often called the radiation paradox. Do you think this a good name?



% \printbibliography
% \appendix



% \section*{Appendix}

% \lstinputlisting[caption = {Python code}, basicstyle = \tiny]{../../Assignment1.py}

\end{document}
